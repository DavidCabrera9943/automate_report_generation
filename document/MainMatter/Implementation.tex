\chapter{Validación de la Implementación y Experimentos}\label{chapter:implementation}

En este capítulo se detalla la implementación de la solución analizando un caso de prueba paso a paso, describiendo en cada uno de estos el proceso seguido al detalle. Para este caso de prueba estaremos usando el modelo Llama3.3-70b debido a que la experiencia en pruebas previas ha mostrado que este modelo es el que mejor tiempo de respuesta tiene.

Para este caso estaremos utilizando un dataset llamado \href{https://huggingface.co/datasets/sayanroy058/Business-Sales/viewer}{\textbf{Business-Sales}}, el cual representa datos de transacciones de ventas de automóviles. Este dataset resulta ideal para evaluar la capacidad del modelo Llama3.3-70b, tanto en la generación de informes automáticos como en la resolución de consultas en lenguaje natural.

\subsubsection{Consideraciones Iniciales sobre el Dataset}
Antes de describir las características principales del dataset, es fundamental destacar algunos aspectos críticos que afectan su procesamiento y análisis:
\begin{itemize}
	\item{Datos Faltantes:}
	Existen valores faltantes en algunas columnas, lo que requiere que el LLM analice e implemente alguna tecnica de imputacion para completar los valores faltantes.
	
	\item{Columnas Irrelevantes:}
	Algunas columnas, como VIN, contienen información única para cada vehículo y no aportan valor para los análisis o reportes generales. Por lo tanto, el modelo deberia darse cuenta de esto y ya sea desecharla en el preprocesamiento o por lo menos no utilizarla a la hora de recibir informacion.
	
	\item{Columnas con Abreviaturas:}
	Existen columnas como MMR (Manheim Market Report) cuyo significado no es explícito. El modelo deberá razonar en contexto para comprender su relevancia y cómo utilizarla en el análisis.
	
	\item{Interpretación de Rangos:}
	Algunas columnas, como condition, presentan valores numéricos cuya interpretación no es inmediata, ell modelo debera inferir sobre el rango y significado de estos valores para obtener resultados adecuados.
	
	\item{Razonamiento sobre Datos Heterogéneos:}
	Las columnas combinan datos categóricos, numéricos y textuales, lo que pone a prueba la capacidad del modelo para generalizar y extraer patrones útiles.
\end{itemize}
