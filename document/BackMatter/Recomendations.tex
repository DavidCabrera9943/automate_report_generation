\begin{recomendations}
    A partir de las conclusiones de esta tesis, se proponen las siguientes recomendaciones para futuras investigaciones y desarrollos:
    
    Se recomienda realizar una evaluación sistemática y comparativa de diversos modelos LLM, tanto open-source como propietarios, en la tarea específica de generación de reportes. Esta evaluación debería considerar métricas cuantitativas (precisión, coherencia, completitud) y cualitativas (utilidad, claridad, relevancia). Asi como la posibilidad de utilizar fine-tunning para tener un modelo mas robusto en la generacion de reportes.
    
    Una posible mejora que se plantea en casi todos los programas que  utilizan skeleton-of-thougt como estructura principal es la de paralelizacion, es decir, ir generando cada seccion en un hilo o proceso de trabajo distinto lo que permitiria reducir el tiempo de generacion de los reportes. En este trabajo no se abarco esa posibilidad debido a las limitaciones de la API.
    
    Es fundamental continuar investigando y desarrollando técnicas de prompt engineering más avanzadas y sofisticadas. Se sugiere explorar estrategias como el aprendizaje few-shot más elaborado y el chain-of-thought prompting para razonamiento más complejo.

    Un paso evolutivo importante sería extender el sistema para que pueda interactuar directamente con bases de datos relacionales, NoSQL, APIs y otras fuentes de datos externas. Esto requeriría desarrollar mecanismos para la conexión segura a estas fuentes, la generación de consultas en diferentes lenguajes (SQL, GraphQL, etc.) y la integración fluida de los resultados en el proceso de generación de reportes.
    
    Se recomienda realizar estudios de caso en dominios específicos (finanzas, medicina, ingeniería, etc.) para evaluar la efectividad del sistema en escenarios prácticos y relevantes. Es crucial llevar a cabo evaluaciones con usuarios reales (analistas de datos, profesionales de negocios, etc.) para medir la usabilidad, utilidad y el impacto del sistema en entornos de trabajo reales. Recopilar feedback de los usuarios sería esencial para identificar áreas de mejora y refinar el diseño del sistema para satisfacer las necesidades reales.
    
    Dado el potencial de los LLMs para generar contenido sesgado o incorrecto, se recomienda investigar activamente estrategias para mitigar estos problemas en el contexto de la generación de reportes. Esto podría incluir técnicas de fact-checking automatizado, el uso de bases de conocimiento verificadas, y el desarrollo de prompts que fomenten la objetividad y la precisión.
    
    Implementar estas recomendaciones permitiría avanzar significativamente en el desarrollo de sistemas de generación automatizada de reportes más potentes, confiables y adaptados a las necesidades de diversos usuarios y dominios.
\end{recomendations}
